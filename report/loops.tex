\section{Loops and Structure of the program}
\subsection{Entry Point}
This loops acts as the entry-point for the program, encompassing close to all of the performed computation. It is not parallel in its
 current form. The variables \verb!strike! and \verb!globs! are both written to and read from in each iteration causing cross-iteration
 dependencies. However, they are both immeditely overwritten, \verb!globs! in the first three lines of \verb!value!, so the dependencies
 cannot be true dependencies and the loop is parallisable.
\begin{lstlisting}[caption=Outermost loop, label=outerloop]
REAL strike;
PrivGlobs    globs(numX, numY, numT);

for( unsigned i = 0; i < outer; ++ i ) {
      strike = 0.001*i;
      res[i] = value( globs, s0, strike, t,
                      alpha, nu,    beta,
                      numX,  numY,  numT );
}
\end{lstlisting}

\begin{lstlisting}[caption=Value function, label=value]
initGrid(s0,alpha,nu,t, numX, numY, numT, globs);
initOperator(globs.myX,globs.myDxx);
initOperator(globs.myY,globs.myDyy);

setPayoff(strike, globs);
\end{lstlisting}


\subsection{Second level}

\paragraph{setPayoff}
This 2-dimensional loop found in the \verb!setPayoff! function contains a trivially parallel inner-nest.
 However, the outer nest is not parallel due to a read-after-write
 dependency between the statements on line 3 and 5.
\begin{lstlisting}[caption=setPayoff() loop, label=payoffloop]
for(unsigned i=0;i<globs.myX.size();++i)
{
	REAL payoff = max(globs.myX[i]-strike, (REAL)0.0);
	for(unsigned j=0;j<globs.myY.size();++j)
		globs.myResult[i][j] = payoff;
}
\end{lstlisting}

\paragraph{Time}
This loop is inherently sequential. Each iteration uses reads and writes to arrays that are part \verb!globs! making
 each iteration dependant on the previous.
\begin{lstlisting}[caption=Timeline loop, label=timeloop]
for(int i = globs.myTimeline.size()-2;i>=0;--i)
    {
        updateParams(i,alpha,beta,nu,globs);
        rollback(i, globs);
    }
\end{lstlisting}

\subsection{Third level}
\label{sec:third}

\paragraph{updateParams}
This loop is trivially parallel in both loop nests as it does not write to any of the locations it reads from, and
 doesn't write to the same location twice.
\begin{lstlisting}[caption=Loop in updateParams(), label=updpar]
for(unsigned i=0;i<globs.myX.size();++i)
   for(unsigned j=0;j<globs.myY.size();++j) {
       globs.myVarX[i][j] = exp(2.0*(  beta*log(globs.myX[i])
                                     + globs.myY[j]
                                     - 0.5*nu*nu*globs.myTimeline[g])
                               );
       globs.myVarY[i][j] = exp(2.0*(  alpha*log(globs.myX[i])
                                     + globs.myY[j]
                                     - 0.5*nu*nu*globs.myTimeline[g])
                               );
   }
\end{lstlisting}

\subsubsection{Rollback}

\paragraph{Explicit x and y loops} These loops contains no cross iteration dependencies and are therefore trivially parallel.
\begin{lstlisting}[caption=Explicit x loop, label=exloop]
  for(i=0;i<numX;i++) {
        for(j=0;j<numY;j++) {
            u[j][i] = dtInv*globs.myResult[i][j];

            if(i > 0) {
              u[j][i] += 0.5*( 0.5*globs.myVarX[i][j]
                                  *globs.myDxx[i][0] )
                            * globs.myResult[i-1][j];
            }
            u[j][i]  +=  0.5*( 0.5*globs.myVarX[i][j]
                                  *globs.myDxx[i][1] )
                            * globs.myResult[i][j];
            if(i < numX-1) {
              u[j][i] += 0.5*( 0.5*globs.myVarX[i][j]
                                  *globs.myDxx[i][2] )
                            * globs.myResult[i+1][j];
            }
        }
    }
\end{lstlisting}
\begin{lstlisting}[caption=Explicit y loop, label=eyloop]
    for(j=0;j<numY;j++)
    {
        for(i=0;i<numX;i++) {
            v[i][j] = 0.0;

            if(j > 0) {
              v[i][j] +=  ( 0.5*globs.myVarY[i][j]
                               *globs.myDyy[j][0] )
                         *  globs.myResult[i][j-1];
            }
            v[i][j]  +=   ( 0.5*globs.myVarY[i][j]
                               *globs.myDyy[j][1] )
                         *  globs.myResult[i][j];
            if(j < numY-1) {
              v[i][j] +=  ( 0.5*globs.myVarY[i][j]
                               *globs.myDyy[j][2] )
                         *  globs.myResult[i][j+1];
            }
            u[j][i] += v[i][j];
        }
    }
\end{lstlisting}

\paragraph{Implicit x and y loops} These loops are not parallel yet due to tridag depending on \verb!a!, \verb!b!, and \verb!c!. However,
 it is not inherently sequential.
\begin{lstlisting}[caption=Implicit x loop, label=impxloop]
for(j=0;j<numY;j++) {
    for(i=0;i<numX;i++) {
        a[i] =		 - 0.5*(0.5*globs.myVarX[i][j]*globs.myDxx[i][0]);
        b[i] = dtInv - 0.5*(0.5*globs.myVarX[i][j]*globs.myDxx[i][1]);
        c[i] =		 - 0.5*(0.5*globs.myVarX[i][j]*globs.myDxx[i][2]);
    }
    tridag(a,b,c,u[j],numX,u[j],yy);
}
\end{lstlisting}
\begin{lstlisting}[caption=Implicit y loop, label=impyloop]
for(i=0;i<numX;i++) {
     for(j=0;j<numY;j++) {
         a[j] =		 - 0.5*(0.5*globs.myVarY[i][j]*globs.myDyy[j][0]);
         b[j] = dtInv - 0.5*(0.5*globs.myVarY[i][j]*globs.myDyy[j][1]);
         c[j] =		 - 0.5*(0.5*globs.myVarY[i][j]*globs.myDyy[j][2]);
     }
      for(j=0;j<numY;j++)
         y[j] = dtInv*u[j][i] - 0.5*v[i][j];

     tridag(a,b,c,y,numY,globs.myResult[i],yy);
 }
\end{lstlisting}


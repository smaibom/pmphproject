\section{Transformations}

\subsection{Array Expansion}
To move the outer loop into the timeline loop on listing \ref{timeloop} and make the outer loop parallel, a number of array expansions in the global structure and local arrays in the rollback function is required.\\

Arrays which needs to be expanded outer times: 
The myResults reads the outer value for calculating strike in the set payoff function in listing \ref{payoffloop}.

The temporary arrays u and in rollback depends on the value from myResult, so it has a dependancy on the outer value. 


The a,b and c arrays for x and y is 



The arrays myX, myY, myDxx, myDyy and myTimeline does not need to be array expanded, there is no dependancy on the value of the outer loop. myVarX and myVarY only depends on the timeline loop, which is sequential.







myX, myY and myTimeline are read from in the updateParams in listing \ref{updpar}. myDxx and myDyy are read from in the explicit and implicit x and y functions in listings \ref{exloop}, \ref{eyloop}, \ref{impxloop}, \ref{impyloop}. When they are initialised in the value function in listing \ref{value}, they are not dependant on the outer loop value

The initialisations 

\subsection{Nest interchange and loop distribution}

\subsubsection{Structural}
\paragraph{Interchanging the outer loop inwards}
\subparagraph{Motivation} In the original structure of the program, we have that a whole slew of loops
 (all of the loops in section \ref{sec:third}) are nested
 within the sequential time iterating loop (listing \ref{timeloop}) which itself is positioned within
 the outer loop (listing \ref{outerloop}). Meaning that we in general had 4-tiered loop nests of the following difficult to parallise
 structure.
$$\mathrm{parallel} \to \mathrm{sequential} \to \mathrm{parallel} \to \mathrm{parallel}$$

\subparagraph{Action} Performed a nest interchange on the outer and time loops.

\subparagraph{Result}

\subparagraph{Validity}


\subsubsection{Memory-based}


\subsection{Transposition}


\section{Transformations}

\subsection{Array Expansion}
\label{sec:arrayexp}

\subsection{Nest interchange and loop distribution}

\subsubsection{Structural}
\paragraph{Interchanging the outer loop inwards}
\subparagraph{Motivation:} In the original structure of the program, we have that a whole slew of loops
 (all of the loops in section \ref{sec:third}) are nested
 within the sequential time iterating loop (Listing \ref{timeloop}) which itself is positioned within
 the outer loop (Listing \ref{outerloop}). Meaning that we in general had 4-tiered loop nests of the following
 structure.
$$\mathrm{parallel} \to \mathrm{sequential} \to \mathrm{parallel} \to \mathrm{parallel}$$
We would prefer to have any sequential loops in the outer nests so that we can more effectively
 parallelise.
\subparagraph{Action:} Performed a nest interchange on the outer and time loops, moving the outer loop inwards.
\subparagraph{Result:} New structure where parallel loops are nested within the sequential loop.
$$\mathrm{sequential} \to \mathrm{parallel} \to \mathrm{parallel} \to \mathrm{parallel}$$
\subparagraph{Validity:} Moving a parallel loop inwards is always valid.

\paragraph{Distribution of Implicit x \& y loops}
\subparagraph{Motivation:} Currently the implicit x/y (Listing \ref{impxloop} \& \ref{impyloop}) consist of one or two loops along with a call
 to \verb!tridag!. We would prefer perfect loop nests.
\subparagraph{Action:} Distribute the secondmost inner nest across the innermost nests.
\subparagraph{Result:} Two/three (for x/y) perfect loop nests.
\subparagraph{Validity:} All used arrays are already array expanded as per section \ref{sec:arrayexp}.
\paragraph{Distributing the outer loop}
\subparagraph{Motivation:} Moving the outer loop inwards, leaves us with a big parallel loop nesting smaller double nested parallel loop
 (Figure \ref{fig:bintouter}.
 These inner loops all have different dimensions making it difficult to make a cohesive parallisation across them all. We would much
 prefer mutliple triple-tiered loops with perfect loop nests.\\
 \begin{figure}[h!]
   \centering
\begin{tikzpicture}[every node/.style={draw,rectangle}]
     \node (time) {time}; \node [right= of time] (outer) {outer};
     \node [right= of outer, draw = none] (dot) {$\vdots$}; \node
     [above= 0.1cm of dot] (1) {loop 1}; \node [below= 0.1cm of dot]
     (2) {loop n};

     \path [->] (time) edge (outer) [->] (outer) edge (1) edge (2);
   \end{tikzpicture}
   \caption{Loop nesting before interchange of outer}
   \label{fig:bintouter}
 \end{figure}
\subparagraph{Action:} Distribute the outer loop across its contained loops.
\subparagraph{Result:} All loops that had an outer-dimension are now perfect loop nests with the structure in Figure \ref{fig:aintouter}.
\begin{figure}[h!]
  \centering
\begin{tikzpicture}[every node/.style={draw,rectangle}]
    \node (time) {time}; \node [right= of time, draw = none] (dot)
    {$\vdots$}; \node [above right= 0.1cm of dot] (1) {loop 1}; \node
    [below right= 0.1cm of dot] (2) {loop n}; \node [left= 0.2cm of 2]
    (outer2) {outer}; \node [left= 0.2cm of 1] (outer) {outer};

    \path [->] (time) edge (outer) edge (outer2) [->] (outer) edge (1)
    [->] (outer2) edge (2);
  \end{tikzpicture}
  \caption{Loop nesting after interchange of outer.}
  \label{fig:aintouter}
\end{figure}
\subparagraph{Validity:} This transformation is dependent on some of the already performed array expansions in section \ref{sec:arrayexp}.
 With these expansions, the transformation is valid. It does not change the order of execution within each iteration of the outer loop.
\subsubsection{Memory-motivated}

\paragraph{Loop interchanging for memory coalescence}
\subparagraph{Motivation:} Given that our working set of arrays are all stored in row-major order, we have that subsequent columns in the
 same row are stored subsequently in memory. To take advantage of spatial-locality and CUDA's memory banks we wish to iterate over columns
 in our innermost nests. This is not the case in loops such as the explicit y loop (Listing \ref{eyloop}).
\subparagraph{Action:} Interchange the two innermost loop nests where needed.
\subparagraph{Result:} Mostly improved memory access patterns. However, some variables - such as \verb!myVarX! in (Listing \ref{exloop}) -
 are now accessed uncoalesced.
\subparagraph{Validity:} Parallel loops can always be interchanged inwards.
\paragraph{Merging explicit \& implicit x/y}
\subparagraph{Motivation:} One of the loops resulting from each of the distribution of the implicit x/y loops are now on the form.
\begin{lstlisting}
  for i = 1..n
   for j = 1..m
    a[i][j] = ...
    b[i][j] = ...
    c[i][j] = ...
   endfor
  endfor
\end{lstlisting}
 Each statement of each iteration of these loops reference the same memory location as the earlier explicit x/y
 (Listings \ref{exloop} and \ref{eyloop}) such as to \verb!myVarX! - and now have the same nest structure. We wish make these loads just
 once for each iteration.
\subparagraph{Action:} Merge the explicit x and y loops with the corresponding implicit loop nests. Expand \verb!a!, \verb!b!, and \verb!c!
 to two arrays, i.e. \verb!ax! and \verb!ay!, as they can no longer be reused.
\subparagraph{Result:} Larger x and y loops.
\subparagraph{Validity:} The \verb!a!, \verb!b!, and \verb!c! arrays had been reused, so the validity of the transformation is dependent on
 the expansion of the two arrays.
\subsection{Transposition}


\section{Transformations}

\subsection{Array Expansion}

\subsection{Nest interchange and loop distribution}

\subsubsection{Structural}
\paragraph{Interchanging the outer loop inwards}
\subparagraph{Motivation:} In the original structure of the program, we have that a whole slew of loops
 (all of the loops in section \ref{sec:third}) are nested
 within the sequential time iterating loop (listing \ref{timeloop}) which itself is positioned within
 the outer loop (listing \ref{outerloop}). Meaning that we in general had 4-tiered loop nests of the following
 structure.
$$\mathrm{parallel} \to \mathrm{sequential} \to \mathrm{parallel} \to \mathrm{parallel}$$
We would prefer to have any sequential loops in the outer nests so that we can more effectively
 parallelise.
\subparagraph{Action:} Performed a nest interchange on the outer and time loops, moving the outer loop inwards.
\subparagraph{Result:} New structure where parallel loops are nested within the sequential loop.
$$\mathrm{sequential} \to \mathrm{parallel} \to \mathrm{parallel} \to \mathrm{parallel}$$
\subparagraph{Validity:} Moving a parallel loop inwards is always valid.

\paragraph{Distributing the outer loop}

\paragraph{Motivation:} Moving the outer loop inwards, leaves us with a big parallel loop nesting smaller double nested parallel loop
 (Figure \ref{fig:bintouter}.
 These inner loops all have different dimensions making it difficult to make a cohesive parallisation across them all. We would much
 prefer mutliple triple-tiered loops with perfect loop nests.\\
 \begin{figure}[h!]
   \centering
\begin{tikzpicture}[every node/.style={draw,rectangle}]
     \node (time) {time}; \node [right= of time] (outer) {outer};
     \node [right= of outer, draw = none] (dot) {$\vdots$}; \node
     [above= 0.1cm of dot] (1) {loop 1}; \node [below= 0.1cm of dot]
     (2) {loop n};

     \path [->] (time) edge (outer) [->] (outer) edge (1) edge (2);
   \end{tikzpicture}
   \caption{Loop nesting before interchange of outer}
   \label{fig:bintouter}
 \end{figure}
\subparagraph{Action:} Distribute the outer loop across its contained loops.
\subparagraph{Result:} All loops that had an outer-dimension are now perfect loop nests with the structure in Figure \ref{fig:aintouter}.
\begin{figure}[h!]
  \centering
\begin{tikzpicture}[every node/.style={draw,rectangle}]
    \node (time) {time}; \node [right= of time, draw = none] (dot)
    {$\vdots$}; \node [above right= 0.1cm of dot] (1) {loop 1}; \node
    [below right= 0.1cm of dot] (2) {loop n}; \node [left= 0.2cm of 2]
    (outer2) {outer}; \node [left= 0.2cm of 1] (outer) {outer};

    \path [->] (time) edge (outer) edge (outer2) [->] (outer) edge (1)
    [->] (outer2) edge (2);
  \end{tikzpicture}
  \caption{Loop nesting after interchange of outer.}
  \label{fig:aintouter}
\end{figure}

\subparagraph{Validity:}

\subsubsection{Memory-motivated}

\subsection{Transposition}

